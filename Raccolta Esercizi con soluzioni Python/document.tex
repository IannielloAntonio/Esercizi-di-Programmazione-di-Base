\documentclass[a4paper,12pt]{article}

% Lingua e codifica
\usepackage[utf8]{inputenc}
\usepackage[T1]{fontenc}
\usepackage[italian]{babel}
\usepackage[a4paper, left=2cm, right=2cm, top=2cm, bottom=2cm]{geometry}

% Impostazioni grafiche e codice
\usepackage{listings}
\usepackage{xcolor}
\usepackage{amsmath}

% Stile per il codice
\lstset{
	language=Python,
	backgroundcolor=\color{gray!10},
	basicstyle=\ttfamily\small,
	frame=single,
	breaklines=true,
	keywordstyle=\color{blue},
	commentstyle=\color{gray!70},
	stringstyle=\color{orange},
	showstringspaces=false
}

\title{Esercizi di Programmazione}
\author{Antonio Ianniello}
\date{}   % <- rimuove la data

\begin{document}
	
	\maketitle
	\newpage
	\tableofcontents
	
	\newpage
	
	\section{Esercizi}
	
	\subsection{Sequenza}
	
	\subsubsection{Quadrato e cubo}
	Scrivere un programma che chieda all'utente di inserire un numero e calcoli sia il quadrato sia il cubo del numero. Stampare i risultati.
	
	\subsubsection{Conversione minuti in ore e minuti}
	Scrivere un programma che chieda all'utente un numero di minuti e calcoli quante ore e quanti minuti residui corrispondono. Stampare il risultato.
	
	\subsubsection{Conversione temperature}
	Scrivere un programma che chieda all'utente una temperatura in gradi Celsius. Calcolare la temperatura equivalente in gradi Fahrenheit e Kelvin. Stampare i risultati.
	
	\subsubsection{Somma, differenza e prodotto}
	Scrivere un programma che chieda all'utente due numeri e calcoli la somma, la differenza, il prodotto e il quoziente. Stampare tutti i risultati.
	
	
	\subsubsection{Area e perimetro di un rettangolo}
	Chiedere all'utente la base e l'altezza di un rettangolo. Calcolare l'area e il perimetro e stampare i risultati.
	
	\subsubsection{Area e circonferenza di un cerchio}
	Chiedere all'utente il raggio di un cerchio e calcolare l'area e la circonferenza. Stampare i risultati.
	
	\subsubsection{Prezzo totale con IVA}
	Chiedere all'utente il prezzo netto di un prodotto e la percentuale di IVA. Calcolare l'importo IVA e il prezzo totale e stamparli.
	
	\subsubsection{Sconto}
	Chiedere all'utente il prezzo pieno di un prodotto e la percentuale di sconto da applicare. Calcolare l'importo dello sconto e il prezzo finale scontato. Stampare tutti i risultati.
	

	
	\subsubsection{Parallelepipedo}
	Chiedere all'utente le dimensioni di un parallelepipedo (lunghezza, larghezza, altezza). Calcolare il volume e l'area totale delle superfici. Stampare i risultati.
	
	\subsubsection{Velocita' media}
	Chiedere all'utente la distanza percorsa e il tempo impiegato. Calcolare la velocita' media e stamparla.
	
	\subsubsection{Caramelle}
	Chiedere all'utente il numero totale di caramelle e il numero di amici. Calcolare quante caramelle ciascun amico riceve (tutte devono ricevere lo stesso numero) e quante caramelle rimangono non distribuite. Stampare i risultati.
	
	
	\subsubsection{Moto Accelerato Uniforme}
	Un’auto parte da ferma e accelera uniformemente lungo una strada. L’accelerazione dell’auto è costante e pari a 2,5 m/s². Dopo 8 secondi, calcolare:
	
	\begin{itemize}
		\item 	La velocità dell’auto.
		\item 	Lo spazio percorso.
		\item 	Il tempo necessario per raggiungere la velocità di 40 m/s.
	\end{itemize}

	
	Scrivere un programma che chieda all’utente di inserire l’accelerazione e il tempo, e che calcoli i valori richiesti.
	


\newpage

\subsection{Selezione}

\subsubsection{Positivo o Negativo}

Scrivere un programma che chieda all’utente di inserire un numero. Il programma deve determinare se il numero inserito sia positivo o negativo e stampare un messaggio che lo comunichi all’utente.
	

\subsubsection{Valore Assoluto}
Scrivere un programma che chieda all’utente di inserire un numero. Il programma deve calcolare il valore assoluto del numero inserito e stampare il risultato.

\subsubsection{Maggiorenne}
Scrivere un programma che chieda all’utente di inserire la propria età. Il programma deve verificare se l’utente è maggiorenne (cioè ha almeno 18 anni) e stampare un messaggio che lo comunichi.


\subsubsection{Multiplo}
Scrivere un programma che legga due numeri dall’utente e verifichi se il primo è multiplo del secondo. Stampare il risultato.
	
	
\subsubsection{Conversione Temperature}
Scrivere un programma che chieda all’utente di scegliere tra due conversioni di temperatura:

\begin{itemize}
	\item Celsius -> Fahrenheit
	\item Fahrenheit -> Celsius
\end{itemize}



Il programma deve:
\begin{itemize}
	\item Leggere la temperatura inserita dall’utente.
	\item Controllare che la temperatura non sia inferiore allo zero assoluto.
	\item Effettuare la conversione richiesta.
	\item Stampare il risultato.
\end{itemize}


\subsubsection{Uomo sulla Luna}
Scrivere un programma che legga dall'utente l'anno di nascita. 
Il programma deve verificare se l'utente e' nato nel 1969. 
Se no, deve calcolare quanti anni prima o dopo il 1969 e stampare il risultato.


\subsubsection{Equazione di Secondo Grado}
Scrivere un programma che legga dall'utente i coefficienti $a$, $b$ e $c$ di un'equazione di secondo grado. 
Il programma deve calcolare le soluzioni reali dell'equazione, se esistono, e stamparle.


	
\subsubsection{Valutazione del voto}
	
	Scrivere un programma che legga un voto dall'utente. 
	Il programma deve indicare se il voto e' insufficiente o sufficiente. 
	Se il voto e' insufficiente, deve distinguere tra gravemente insufficiente (minore o uguale a 4) o insufficiente (compreso tra 4 escluso e 6 escluso).
	
	
\subsubsection{Massimo tra tre numeri}

Scrivere un programma che legga tre numeri dall'utente. 
Il programma deve stampare quale dei tre numeri e' il maggiore: "il maggiore e' il primo", "il maggiore e' il secondo" o "il maggiore e' il terzo".


\subsubsection{Biglietto VIP}

Scrivere un programma che legga l'età e il tipo di biglietto dell'utente. 
Il programma deve stampare "Accesso consentito" se l'utente ha meno di 18 anni oppure possiede un biglietto VIP. 
Altrimenti deve stampare "Accesso negato".
	
	
\subsubsection{Re e Regina}

Su una scacchiera $8 \times 8$ sono posizionati due pezzi: il Re bianco e la Regina nera.  
Scrivere un programma che, acquisite le posizioni del Re e della Regina, determini se la Regina è in posizione tale da poter mangiare il Re.  

Le posizioni dei due pezzi sono identificate mediante la riga e la colonna su cui si trovano, espresse come numeri interi tra 1 e 8.

	
	
\subsection{Iterazioni}

\subsubsection{Primi $n$ numeri}
	Scrivere un programma che chiede all'utente di inserire un numero intero $n$ e stampa i primi $n$ numeri naturali a partire da 1.
	
	
\subsubsection{Somma di N numeri}
Scrivere un programma che chiede all'utente di inserire un numero intero $N$, che rappresenta la quantità di numeri da sommare. Successivamente il programma legge i $N$ numeri inseriti dall'utente e ne calcola la somma totale.

\subsubsection{Numero Positivo}
Scrivere un programma che chiede all'utente di inserire un numero intero positivo $n$. Se l'utente inserisce un numero negativo o nullo, il programma deve continuare a richiedere il valore finché non viene inserito un numero positivo.


\subsubsection{Calcolo della Media}
Scrivere un programma che calcoli la media aritmetica di una serie di numeri inseriti da tastiera.

\begin{itemize}
	\item Nella prima versione, l'inserimento termina quando l'utente digita il valore \texttt{0}. Il programma deve poi calcolare e stampare la media dei valori inseriti (escludendo lo zero).
	\item Nella seconda versione, il programma chiede prima quanti numeri l'utente intende inserire (\texttt{n}) e poi calcola la media dei numeri forniti.
\end{itemize}


\subsubsection{Fattoriale}
Scrivere un programma che acquisisca da tastiera un numero intero positivo \texttt{N} e calcoli il valore del suo fattoriale.

Il fattoriale di un numero è il prodotto di tutti i numeri compresi tra 1 ed \texttt{N}.  
Si ricorda che:
\[
N! = 1 \times 2 \times 3 \times \dots \times N
\]
Inoltre, per convenzione:
\[
0! = 1
\]

\subsubsection{Tabellina}
Scrivere un programma che chieda all'utente un numero intero positivo e stampi la sua tabellina fino a 10.


\subsubsection{Fibonacci}
Scrivere un programma che chieda all'utente un numero intero positivo N e stampi i primi N termini della serie di Fibonacci.
La serie di Fibonacci è una sequenza di numeri tali per cui ognuno di essi corrisponde alla somma dei due precedenti:

\[
1 \quad 2 \quad 3 \quad 5 \quad 8 \quad 13 \quad 21 \quad 34 \quad \dots
\]










\newpage

\subsection{Strutture Miste}
\subsubsection{Divisori di un numero}
Scrivere un programma che, letto un numero intero positivo, visualizzi tutti i suoi divisori.

Si ricorda che il resto della divisione si calcola con l’operatore \texttt{\%}.  
Ad esempio:  
\[
5 \% 2 = 1
\]
	
	
	\subsubsection{Numeri Primi}
	Scrivere un programma che legge un numero intero positivo e verifica se è un numero primo.
	
	Si ricorda che un numero primo è un numero maggiore di 1 che non ha altri divisori oltre a 1 e se stesso.
	
	
	
	\subsubsection{Gioco "Indovina un numero"}
	
	Si scriva un programma che permetta di giocare al gioco "Indovina un numero".  
	
	Un primo utente inserisce da tastiera un numero segreto compreso tra 1 e 100.  
	Il secondo utente deve indovinare il numero entro un massimo di 10 tentativi.  
	
	Ad ogni tentativo il programma stampa:
	- "Esatto!" se il numero è corretto,  
	- "Troppo alto" se il numero ipotizzato è maggiore di quello segreto,  
	- "Troppo basso" se il numero ipotizzato è minore di quello segreto.  
	
	Se il numero non viene indovinato entro 10 tentativi, il programma stampa "Hai perso".
	
	
	
	
	
	
	
	
	
	
	
	
	
	
	
	
	
	
	
	\newpage
	\section{Soluzioni}
	
	\subsection{Sequenza}

	\subsubsection{Quadrato e cubo}
	\begin{lstlisting}
# Chiedo all'utente di inserire un numero
# Uso float per permettere numeri decimali
numero = float(input("Inserisci un numero: "))

# Calcolo il quadrato e il cubo del numero
quadrato = numero ** 2
cubo = numero ** 3
	
# Stampo i risultati
print("Il quadrato di", numero, "e'", quadrato)
print("Il cubo di", numero, "e'", cubo)
	\end{lstlisting}
	

\newpage
	\subsubsection{Conversione minuti in ore e minuti}
	\begin{lstlisting}
# Chiedo all'utente il numero totale di minuti
minuti_totali = int(input("Inserisci il numero di minuti: "))
		
# Calcolo le ore (divisione intera)
ore = minuti_totali // 60
		
# Calcolo i minuti residui (resto della divisione)
minuti = minuti_totali % 60
		
# Stampo i risultati
print("Ore:", ore)
print("Minuti residui:", minuti)
	\end{lstlisting}
	
	\newpage
	\subsubsection{Conversione temperature}
	\begin{lstlisting}
# Chiedo all'utente la temperatura in gradi Celsius
celsius = float(input("Inserisci la temperatura in gradi Celsius: "))
	
# Calcolo Fahrenheit e Kelvin
fahrenheit = celsius * 9/5 + 32
kelvin = celsius + 273.15
		
# Stampo i risultati
print("Temperatura in Fahrenheit:", fahrenheit)
print("Temperatura in Kelvin:", kelvin)
	\end{lstlisting}
	
	\newpage
	\subsubsection{Somma, differenza e prodotto}
	\begin{lstlisting}
# Chiedo all'utente due numeri
num1 = float(input("Inserisci il primo numero: "))
num2 = float(input("Inserisci il secondo numero: "))
		
# Calcolo somma, differenza, prodotto e quoziente
somma = num1 + num2
differenza = num1 - num2
prodotto = num1 * num2
quoziente = num1 / num2  # divisione normale, puo' dare decimali
		
# Stampo i risultati
print("Somma:", somma)
print("Differenza:", differenza)
print("Prodotto:", prodotto)
print("Quoziente:", quoziente)
	\end{lstlisting}
	
	\newpage
	\subsubsection{Area e perimetro di un rettangolo}
	\begin{lstlisting}
# Chiedo all'utente base e altezza
base = float(input("Inserisci la base del rettangolo: "))
altezza = float(input("Inserisci l'altezza del rettangolo: "))
		
# Calcolo area e perimetro
area = base * altezza
perimetro = 2 * (base + altezza)
		
# Stampo i risultati
print("Area:", area)
print("Perimetro:", perimetro)
	\end{lstlisting}
	
	\newpage
	\subsubsection{Area e circonferenza di un cerchio}
	\begin{lstlisting}
import math  # libreria per costante pi greco
		
# Chiedo all'utente il raggio
raggio = float(input("Inserisci il raggio del cerchio: "))
		
# Calcolo area e circonferenza
area = math.pi * raggio**2
circonferenza = 2 * math.pi * raggio
		
# Stampo i risultati
print("Area:", area)
print("Circonferenza:", circonferenza)
	\end{lstlisting}
	
	\newpage
	\subsubsection{Prezzo totale con IVA}
	\begin{lstlisting}
# Chiedo all'utente il prezzo netto
prezzo_netto = float(input("Inserisci il prezzo netto: "))

# Chiedo all'utente la percentuale di IVA
iva = float(input("Inserisci la percentuale di IVA: "))
	
# Calcolo importo IVA e prezzo totale
importo_iva = prezzo_netto * iva / 100
prezzo_totale = prezzo_netto + importo_iva
		
# Stampo i risultati
print("Importo IVA:", round(importo_iva, 2))  # round arrotonda a 2 decimali
print("Prezzo totale:", round(prezzo_totale, 2))
	\end{lstlisting}
	
	\newpage
	\subsubsection{Sconto}
	\begin{lstlisting}
# Chiedo all'utente il prezzo pieno
prezzo_pieno = float(input("Inserisci il prezzo pieno del prodotto: "))
		
# Chiedo la percentuale di sconto
percentuale_sconto = float(input("Inserisci la percentuale di sconto (%): "))

# Calcolo sconto in valore assoluto
sconto = prezzo_pieno * percentuale_sconto / 100

# Calcolo prezzo scontato
prezzo_scontato = prezzo_pieno - sconto

# Stampo i risultati arrotondati a 2 decimali
print("Prezzo pieno: euro", round(prezzo_pieno, 2))
print("Sconto: euro", round(sconto, 2))
print("Prezzo scontato: euro", round(prezzo_scontato, 2))
	\end{lstlisting}
	

	\newpage
	\subsubsection{Parallelepipedo}
	\begin{lstlisting}
# Chiedo all'utente le dimensioni
lunghezza = float(input("Inserisci la lunghezza del parallelepipedo: "))
larghezza = float(input("Inserisci la larghezza del parallelepipedo: "))
altezza = float(input("Inserisci l'altezza del parallelepipedo: "))
		
# Calcolo volume
volume = lunghezza * larghezza * altezza
		
# Calcolo area totale delle superfici
area_superfici = 2 * (lunghezza*larghezza + lunghezza*altezza + larghezza*altezza)
		
# Stampo risultati
print("Volume:", volume)
print("Area totale delle superfici:", area_superfici)
	\end{lstlisting}
	
	\newpage
	\subsubsection{Velocita' media}
	\begin{lstlisting}
# Chiedo distanza e tempo
distanza = float(input("Inserisci la distanza percorsa (km): "))
tempo = float(input("Inserisci il tempo impiegato (ore): "))
		
# Calcolo velocita' media
velocita = distanza / tempo
		
# Stampo il risultato
print("La velocita' media e':", round(velocita, 2), "km/h")
	\end{lstlisting}
	
	\newpage
	\subsubsection{Caramelle}
	\begin{lstlisting}
# Chiedo numero totale di caramelle
totale_caramelle = int(input("Inserisci il numero totale di caramelle: "))
	
# Chiedo numero di amici
numero_amici = int(input("Inserisci il numero di amici: "))
		
# Calcolo quante caramelle riceve ciascun amico
caramelle_per_amico = totale_caramelle // numero_amici  # divisione intera
		
# Calcolo quante caramelle rimangono
caramelle_rimanenti = totale_caramelle % numero_amici  # resto della divisione
		
# Stampo risultati
print("Ogni amico riceve:", caramelle_per_amico, "caramelle")
print("Rimangono non distribuite:", caramelle_rimanenti, "caramelle")
	\end{lstlisting}
	
	
	\newpage
	\subsubsection{Moto Accelerato Uniforme}
	
	\begin{lstlisting}
# Chiedo all'utente l'accelerazione (in m/s^2)
accelerazione = float(input("Inserisci l'accelerazione (m/s^2): "))

# Chiedo il tempo trascorso (in secondi)
tempo = float(input("Inserisci il tempo trascorso (s): "))

# 1. Calcolo la velocita' finale usando v = a * t
velocita = accelerazione * tempo

# 2. Calcolo lo spazio percorso usando s = 0.5 * a * t^2
spazio = 0.5 * accelerazione * tempo**2

# 3. Calcolo il tempo necessario per raggiungere una velocita' target
velocita_target = float(input("Inserisci la velocita' target (m/s): "))
tempo_target = velocita_target / accelerazione  # t = v/a

# Stampo i risultati arrotondati a 2 decimali
print("Velocita' finale:", round(velocita, 2), "m/s")
print("Spazio percorso:", round(spazio, 2), "m")
print("Tempo per raggiungere", velocita_target, "m/s:", round(tempo_target, 2), "s")

	\end{lstlisting}
	
	
	
	

\newpage
\subsection{Selezione}

\subsubsection{Positivo o Negativo}
VERSIONE CORRETTA
\begin{lstlisting}
# Chiedo all'utente di inserire un numero
numero = float(input("Inserisci un numero: "))

# Le istruzioni dentro if, elif ed else devono essere indentate (tabulate o 4 spazi)
# In Python la tabulazione e' fondamentale per indicare quali istruzioni appartengono
# a ciascun blocco di codice. Senza di essa il programma da errore.
if numero > 0:
	print("Il numero e' positivo")
elif numero < 0:
	print("Il numero e' negativo")
else:
	print("Il numero e' zero")
	
\end{lstlisting}

VERSIONE ERRATA:

\begin{lstlisting}
numero = float(input("Inserisci un numero: "))

# Questa versione e' errata
if numero > 0:
print("Il numero e' positivo")
elif numero < 0:
print("Il numero e' negativo")
else:
print("Il numero e' zero")
\end{lstlisting}

\newpage
\subsubsection{Valore Assoluto}

\begin{lstlisting}
# Chiedo all'utente di inserire un numero
numero = float(input("Inserisci un numero: "))

# Calcolo il valore assoluto usando if
if numero < 0:
	valore_assoluto = -numero  
	# se il numero e' negativo, cambio segno
else:
	valore_assoluto = numero  
	 # se il numero e' positivo o zero, rimane uguale

# Stampo il risultato
print("Il valore assoluto di", numero, "e':", valore_assoluto)
\end{lstlisting}


\newpage
\subsubsection{Maggiorenne}

\begin{lstlisting}
# Chiedo all'utente di inserire la propria eta'
eta = int(input("Inserisci la tua eta': "))

# Verifico se l'utente e' maggiorenne
if eta >= 18:
	print("Sei maggiorenne")
else:
	print("Non sei maggiorenne")
\end{lstlisting}


\newpage
\subsubsection{Multiplo}
\begin{lstlisting}
# Chiedo all'utente di inserire il primo numero
num1 = int(input("Inserisci il primo numero: "))

# Chiedo all'utente di inserire il secondo numero
num2 = int(input("Inserisci il secondo numero: "))

# Verifico se il primo numero e' multiplo del secondo
# Per la verifica vado a verificare il resto della divisione di num1 e num2. Se questo e' 0 allora sono multipli
if num1 % num2 == 0:
	print(num1, "e' multiplo di", num2)
else:
	print(num1, "non e' multiplo di", num2)
\end{lstlisting}

\newpage
\subsubsection{Conversione Temperature}

\begin{lstlisting}
# Chiedo all'utente quale conversione vuole fare
print("1: Celsius -> Fahrenheit")
print("2: Fahrenheit -> Celsius")
scelta = int(input("Inserisci 1 o 2: "))

if scelta == 1:
	celsius = float(input("Inserisci la temperatura in Celsius: "))

	# Controllo che non sia sotto lo zero assoluto
	if celsius < -273.15:
		print("Errore: temperatura sotto lo zero assoluto")
	else:
		fahrenheit = (9/5) * celsius + 32
		kelvin = celsius + 273.15
		print("Fahrenheit:", round(fahrenheit,2))
		print("Kelvin:", round(kelvin,2))

elif scelta == 2:
	fahrenheit = float(input("Inserisci la temperatura in Fahrenheit: "))
	celsius = (fahrenheit - 32) * 5/9

	# Controllo che non sia sotto lo zero assoluto in Celsius
	if celsius < -273.15:
		print("Errore: temperatura sotto lo zero assoluto")
	else:
		kelvin = celsius + 273.15
		print("Celsius:", round(celsius,2))
		print("Kelvin:", round(kelvin,2))

else:
	print("Scelta non valida")

\end{lstlisting}


\newpage

\subsubsection{Uomo sulla Luna}
\begin{lstlisting}
# Anno in cui l'uomo e' andato sulla Luna
anno_luna = 1969

# Chiedo all'utente di inserire l'anno di nascita
anno_nascita = int(input("Inserisci il tuo anno di nascita: "))

# Confronto con l'anno della Luna
if anno_nascita == anno_luna:
	print("Sei nato nell'anno in cui l'uomo e' andato sulla Luna")
elif anno_nascita < anno_luna:
	differenza = anno_luna - anno_nascita
	print("Sei nato", differenza, "anni prima del 1969")
else:
	differenza = anno_nascita - anno_luna
	print("Sei nato", differenza, "anni dopo il 1969")

\end{lstlisting}


\newpage
\subsubsection{Equazioni di Secondo Grado}
\begin{lstlisting}
# Importo il modulo math per usare funzioni matematiche avanzate,
# come sqrt() per calcolare la radice quadrata
import math

# Chiedo i coefficienti all'utente
a = float(input("Inserisci il coefficiente a: "))
b = float(input("Inserisci il coefficiente b: "))
c = float(input("Inserisci il coefficiente c: "))

# Controllo se a e' zero (in tal caso non e' piu' un'equazione di secondo grado)
if a == 0:
	print("Non e' un'equazione di secondo grado")
else:
	# Calcolo del discriminante delta = b^2 - 4*a*c
	delta = b**2 - 4*a*c

	if delta < 0:
		# Se delta < 0 non ci sono soluzioni reali
		print("L'equazione non ha soluzioni reali")
	elif delta == 0:
		# Se delta = 0 c'e' una soluzione reale doppia
		x = -b / (2*a)
		print("L'equazione ha una soluzione reale:", x)
	else:
		# Se delta > 0 ci sono due soluzioni reali distinte
		# Uso math.sqrt(delta) per calcolare la radice quadrata di delta
		x1 = (-b + math.sqrt(delta)) / (2*a)
		x2 = (-b - math.sqrt(delta)) / (2*a)
		print("L'equazione ha due soluzioni reali:", x1, "e", x2)
\end{lstlisting}



\newpage
\subsubsection{Valutazione del voto}
\begin{lstlisting}
# Chiedo all'utente di inserire il voto
voto = float(input("Inserisci il voto: "))

# Verifico se il voto e' sufficiente o insufficiente
if voto >= 6:
	print("Voto sufficiente")
else:
# Se il voto e' insufficiente, distinguo tra gravemente insufficiente o insufficiente
	if voto <= 4:
		print("Gravemente insufficiente")
	else:
		print("Insufficiente")

\end{lstlisting}


\newpage
\subsubsection{Massimo tra tre numeri}

\begin{lstlisting}
# Chiedo all'utente di inserire tre numeri
num1 = float(input("Inserisci il primo numero: "))
num2 = float(input("Inserisci il secondo numero: "))
num3 = float(input("Inserisci il terzo numero: "))

# Confronto i numeri per determinare il maggiore
# L'operatore 'and' serve a verificare che entrambe le condizioni siano vere
# Se anche UNA SOLA delle due e' falsa, la condizione viene reputata falsa
if num1 >= num2 and num1 >= num3:
	print("Il maggiore e' il primo")
elif num2 >= num1 and num2 >= num3:
	print("Il maggiore e' il secondo")
else:
	print("Il maggiore e' il terzo")

\end{lstlisting}




\newpage
\subsubsection{Biglietto VIP}
\begin{lstlisting}
# Chiedo all'utente l'eta'
eta = int(input("Inserisci la tua eta': "))

# Chiedo all'utente il tipo di biglietto (VIP = 1, normale = 0)
biglietto_vip = int(input("Possiedi un biglietto VIP? (1 = si, 0 = no): "))

# Controllo se l'accesso e' consentito
# Uso 'or' per verificare due condizioni alternative
# L'accesso e' consentito se eta < 18 oppure se ha il biglietto VIP
# Con "or" basta che una sola delle due condizioni sia vera per reputare l'intera condizione dell'if vera.
if eta < 18 or biglietto_vip == 1:
	print("Accesso consentito")
else:
	print("Accesso negato")

\end{lstlisting}


\newpage
\subsubsection{Re e Regina}
\begin{lstlisting}
# Acquisisco la posizione del Re
riga_re = int(input("Inserisci la riga del Re (1-8): "))
colonna_re = int(input("Inserisci la colonna del Re (1-8): "))

# Acquisisco la posizione della Regina
riga_regina = int(input("Inserisci la riga della Regina (1-8): "))
colonna_regina = int(input("Inserisci la colonna della Regina (1-8): "))

# Controllo se la Regina puo' mangiare il Re
# La Regina puo' mangiare se e' sulla stessa riga, colonna o diagonale del Re
	if riga_re == riga_regina or colonna_re == colonna_regina or abs(riga_re - riga_regina) == abs(colonna_re - colonna_regina):
	print("La Regina puo' mangiare il Re.")
else:
	print("La Regina non puo' mangiare il Re.")

# Nota:
# Uso abs() per ottenere il valore assoluto della differenza tra righe e colonne.
# Se le differenze tra riga e colonna sono uguali, i due pezzi si trovano sulla stessa diagonale.

\end{lstlisting}

\newpage
\subsection{Iterazioni}



\subsubsection{Primi $n$ numeri}
\begin{lstlisting}
# Chiedo all'utente quanti numeri stampare
n = int(input("Inserisci un numero intero n: "))

# Stampo i numeri da 1 fino a n
# Uso la funzione range(1, n+1) perche' range esclude il limite superiore
for i in range(1, n + 1):
	print(i)

\end{lstlisting}


\newpage
\subsubsection{Somma di N numeri}
\begin{lstlisting}
# Chiedo all'utente quanti numeri vuole sommare
N = int(input("Quanti numeri vuoi sommare? "))

# Inizializzo la variabile somma a 0
somma = 0

# Eseguo un ciclo che si ripete N volte
for i in range(N):
	# Ad ogni iterazione chiedo un numero all'utente
	numero = float(input("Inserisci un numero: "))
	# Aggiungo il numero alla somma totale
	somma += numero

# Alla fine del ciclo stampo la somma complessiva
print("La somma dei numeri inseriti e':", somma)

\end{lstlisting}


\newpage
\subsubsection{Numero Positivo}
\begin{lstlisting}
# Chiedo all'utente di inserire un numero positivo
n = int(input("Inserisci un numero positivo: "))

# Continuo a chiedere il numero finche' non e' positivo
while n <= 0:
	print("Errore: il numero deve essere positivo.")
	n = int(input("Inserisci di nuovo un numero positivo: "))

# Quando il numero e' valido, lo comunico all'utente
print("Hai inserito un numero positivo valido:", n)

\end{lstlisting}


\newpage
\subsubsection{Calcolo della Media}
VERSIONE CON CICLO WHILE
\begin{lstlisting}
# Programma per calcolare la media aritmetica
# Inserendo 0 si termina l'inserimento dei numeri

somma = 0       # Variabile per accumulare la somma dei numeri
conteggio = 0   # Variabile per contare quanti numeri sono stati inseriti

# Chiedo all'utente di inserire un numero
numero = float(input("Inserisci un numero (0 per terminare): "))

# Continuo finche' il numero non e' 0
while numero != 0:
	somma += numero
	conteggio += 1
	numero = float(input("Inserisci un numero (0 per terminare): "))

# Calcolo e stampo la media solo se sono stati inseriti numeri validi
if conteggio > 0:
	media = somma / conteggio
	print("La media dei numeri inseriti e':", media)
else:
	print("Nessun numero inserito, impossibile calcolare la media.")
\end{lstlisting}

VERSIONE CON CICLO FOR
\begin{lstlisting}
# Programma per calcolare la media aritmetica
# L'utente decide quanti numeri inserire

# Chiedo quanti numeri inserira' l'utente
n = int(input("Quanti numeri vuoi inserire? "))

somma = 0

# Uso un ciclo for per leggere n numeri
for i in range(n):
	numero = float(input(f"Inserisci il numero {i+1}: "))
	somma += numero

# Calcolo e stampo la media
if n > 0:
	media = somma / n
	print("La media dei numeri inseriti e': ", media)
else:
	print("Numero di elementi non valido.")

\end{lstlisting}

\newpage
\subsubsection{Fattoriale}

\begin{lstlisting}
# Programma per calcolare il fattoriale di un numero

# Chiedo all'utente di inserire un numero intero positivo
N = int(input("Inserisci un numero intero positivo: "))

# Controllo che il numero sia positivo
# Se non lo e', continuo a chiederlo
while N < 0:
	N = int(input("Il numero deve essere positivo. Riprova: "))

# Inizializzo la variabile fattoriale a 1
# (perche' 0! e 1! valgono 1)
fattoriale = 1

# Uso un ciclo for per moltiplicare tutti i numeri da 1 a N
for i in range(1, N + 1):
	fattoriale *= i
	# Questo e' come dire fattoriale = fattoriale * i
	# Al primo giro fara' fattoriale = 1 * 1
	# Al secondo giro fara' fattoriale = 1 * 2
	# Al terzo giro fara' fattoriale = 2 * 3
	# Al quarto giro fara' fattoriale = 6 * 4 e cosi' via
	
# Stampo il risultato finale
print("Il fattoriale di", N, "e':", fattoriale)

\end{lstlisting}

\newpage
\subsubsection{Tabellina}

\begin{lstlisting}
# Chiedo all'utente di inserire un numero intero positivo
n = int(input("Inserisci un numero intero positivo: "))

# Ciclo per calcolare e stampare i prodotti da 1 a 10
for i in range(1, 11):
	print(n, "x", i, "=", n * i)

\end{lstlisting}


\subsubsection{Fibonacci}

\begin{lstlisting}
n = int(input("Inserire n: "))

a = 1
b = 1

for i in range(n):
	temp = a + b
	a = b
	b = temp
	print(a, end=" ")

\end{lstlisting}











\newpage
\subsection{Strutture Miste}

\subsubsection{Divisori di un numero}

\begin{lstlisting}
# Programma che calcola tutti i divisori di un numero

# Chiedo all'utente di inserire un numero intero positivo
n = int(input("Inserisci un numero intero positivo: "))

# Controllo che il numero sia positivo
while n <= 0:
	n = int(input("Il numero deve essere positivo. Riprova: "))

print("I divisori di", n, "sono:")

# Un divisore di n e' un numero i tale che n % i == 0
for i in range(1, n + 1):
	if n % i == 0:
	print(i)

\end{lstlisting}

\newpage
\subsubsection{Numeri Primi}
\begin{lstlisting}
# Programma che verifica se un numero e' primo
	
# Chiedo all'utente di inserire un numero intero positivo
n = int(input("Inserisci un numero intero positivo: "))

# Controllo che il numero sia positivo
while n <= 0:
	n = int(input("Il numero deve essere positivo. Riprova: "))
	
# Controllo se il numero e' maggiore di 1
if n == 1:
	print("1 non e' un numero primo")
else:
	primo = True  # assumiamo che il numero sia primo

for i in range(2, n):  # verifico i possibili divisori da 2 a n-1
	if n % i == 0:
		primo = False
		break  # basta trovare un divisore per sapere che non e' primo
	
# Stampo il risultato
if primo: # Equivalente a scrivere if primo == True
	print(n, "e' un numero primo")
else:
	print(n, "non e' un numero primo")
	
\end{lstlisting}

\newpage
\subsubsection{Gioco "Indovina un numero"}

\begin{lstlisting}
# Gioco "Indovina un numero"

# Primo utente inserisce il numero segreto
numero_segreto = int(input("Inserisci il numero segreto (tra 1 e 100): "))

print("Indovina il numero!")

# Inizializzo il numero di tentativi
tentativo = 1
max_tentativi = 10
indovinato = False  # Flag per sapere se il numero e' stato indovinato

# Ciclo while finche' non si superano i tentativi massimi e non si indovina
while tentativo <= max_tentativi and not indovinato:
	# Secondo utente inserisce il numero ipotizzato
	ipotesi = int(input(f"Tentativo numero {tentativo}: "))

	# Controllo se il numero e' corretto
	if ipotesi == numero_segreto:
		print("Esatto!")
		indovinato = True
	elif ipotesi < numero_segreto:
		print("Troppo basso")
	else:
		print("Troppo alto")

	tentativo += 1  # Passo al tentativo successivo

# Se non e' stato indovinato entro 10 tentativi
if not indovinato: # Potevo anche scrivere if indovinato == False
	print("Hai perso")


\end{lstlisting}
	
\end{document}
