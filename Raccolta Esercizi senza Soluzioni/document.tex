\documentclass[a4paper,12pt]{article}

% Lingua e codifica
\usepackage[utf8]{inputenc}
\usepackage[T1]{fontenc}
\usepackage[italian]{babel}
\usepackage[a4paper, left=2cm, right=2cm, top=2cm, bottom=2cm]{geometry}

% Impostazioni grafiche e codice
\usepackage{listings}
\usepackage{xcolor}
\usepackage{amsmath}

% Stile per il codice
\lstset{
	language=Python,
	backgroundcolor=\color{gray!10},
	basicstyle=\ttfamily\small,
	frame=single,
	breaklines=true,
	keywordstyle=\color{blue},
	commentstyle=\color{gray!70},
	stringstyle=\color{orange},
	showstringspaces=false
}

\title{Esercizi di Programmazione}
\author{Antonio Ianniello}
\date{}   % <- rimuove la data

\begin{document}
	
	\maketitle
	\newpage
	\tableofcontents
	
	\newpage
	
	\section{Esercizi}
	
	\subsection{Sequenza}
	
	\subsubsection{Quadrato e cubo}
	Scrivere un programma che chieda all'utente di inserire un numero e calcoli sia il quadrato sia il cubo del numero. Stampare i risultati.
	
	\subsubsection{Conversione minuti in ore e minuti}
	Scrivere un programma che chieda all'utente un numero di minuti e calcoli quante ore e quanti minuti residui corrispondono. Stampare il risultato.
	
	\subsubsection{Conversione temperature}
	Scrivere un programma che chieda all'utente una temperatura in gradi Celsius. Calcolare la temperatura equivalente in gradi Fahrenheit e Kelvin. Stampare i risultati.
	
	\subsubsection{Somma, differenza e prodotto}
	Scrivere un programma che chieda all'utente due numeri e calcoli la somma, la differenza, il prodotto e il quoziente. Stampare tutti i risultati.
	
	
	\subsubsection{Area e perimetro di un rettangolo}
	Chiedere all'utente la base e l'altezza di un rettangolo. Calcolare l'area e il perimetro e stampare i risultati.
	
	\subsubsection{Area e circonferenza di un cerchio}
	Chiedere all'utente il raggio di un cerchio e calcolare l'area e la circonferenza. Stampare i risultati.
	
	\subsubsection{Prezzo totale con IVA}
	Chiedere all'utente il prezzo netto di un prodotto e la percentuale di IVA. Calcolare l'importo IVA e il prezzo totale e stamparli.
	
	\subsubsection{Sconto}
	Chiedere all'utente il prezzo pieno di un prodotto e la percentuale di sconto da applicare. Calcolare l'importo dello sconto e il prezzo finale scontato. Stampare tutti i risultati.
	

	
	\subsubsection{Parallelepipedo}
	Chiedere all'utente le dimensioni di un parallelepipedo (lunghezza, larghezza, altezza). Calcolare il volume e l'area totale delle superfici. Stampare i risultati.
	
	\subsubsection{Velocita' media}
	Chiedere all'utente la distanza percorsa e il tempo impiegato. Calcolare la velocita' media e stamparla.
	
	\subsubsection{Caramelle}
	Chiedere all'utente il numero totale di caramelle e il numero di amici. Calcolare quante caramelle ciascun amico riceve (tutte devono ricevere lo stesso numero) e quante caramelle rimangono non distribuite. Stampare i risultati.
	
	
	\subsubsection{Moto Accelerato Uniforme}
	Un’auto parte da ferma e accelera uniformemente lungo una strada. L’accelerazione dell’auto è costante e pari a 2,5 m/s². Dopo 8 secondi, calcolare:
	
	\begin{itemize}
		\item 	La velocità dell’auto.
		\item 	Lo spazio percorso.
		\item 	Il tempo necessario per raggiungere la velocità di 40 m/s.
	\end{itemize}

	
	Scrivere un programma che chieda all’utente di inserire l’accelerazione e il tempo, e che calcoli i valori richiesti.
	


\newpage

\subsection{Selezione}

\subsubsection{Positivo o Negativo}

Scrivere un programma che chieda all’utente di inserire un numero. Il programma deve determinare se il numero inserito sia positivo o negativo e stampare un messaggio che lo comunichi all’utente.
	

\subsubsection{Valore Assoluto}
Scrivere un programma che chieda all’utente di inserire un numero. Il programma deve calcolare il valore assoluto del numero inserito e stampare il risultato.

\subsubsection{Maggiorenne}
Scrivere un programma che chieda all’utente di inserire la propria età. Il programma deve verificare se l’utente è maggiorenne (cioè ha almeno 18 anni) e stampare un messaggio che lo comunichi.


\subsubsection{Multiplo}
Scrivere un programma che legga due numeri dall’utente e verifichi se il primo è multiplo del secondo. Stampare il risultato.
	
	
\subsubsection{Conversione Temperature}
Scrivere un programma che chieda all’utente di scegliere tra due conversioni di temperatura:

\begin{itemize}
	\item Celsius -> Fahrenheit
	\item Fahrenheit -> Celsius
\end{itemize}



Il programma deve:
\begin{itemize}
	\item Leggere la temperatura inserita dall’utente.
	\item Controllare che la temperatura non sia inferiore allo zero assoluto.
	\item Effettuare la conversione richiesta.
	\item Stampare il risultato.
\end{itemize}


\subsubsection{Uomo sulla Luna}
Scrivere un programma che legga dall'utente l'anno di nascita. 
Il programma deve verificare se l'utente e' nato nel 1969. 
Se no, deve calcolare quanti anni prima o dopo il 1969 e stampare il risultato.


\subsubsection{Equazione di Secondo Grado}
Scrivere un programma che legga dall'utente i coefficienti $a$, $b$ e $c$ di un'equazione di secondo grado. 
Il programma deve calcolare le soluzioni reali dell'equazione, se esistono, e stamparle.


	
\subsubsection{Valutazione del voto}
	
	Scrivere un programma che legga un voto dall'utente. 
	Il programma deve indicare se il voto e' insufficiente o sufficiente. 
	Se il voto e' insufficiente, deve distinguere tra gravemente insufficiente (minore o uguale a 4) o insufficiente (compreso tra 4 escluso e 6 escluso).
	
	
\subsubsection{Massimo tra tre numeri}

Scrivere un programma che legga tre numeri dall'utente. 
Il programma deve stampare quale dei tre numeri e' il maggiore: "il maggiore e' il primo", "il maggiore e' il secondo" o "il maggiore e' il terzo".


\subsubsection{Biglietto VIP}

Scrivere un programma che legga l'età e il tipo di biglietto dell'utente. 
Il programma deve stampare "Accesso consentito" se l'utente ha meno di 18 anni oppure possiede un biglietto VIP. 
Altrimenti deve stampare "Accesso negato".
	
	
\subsubsection{Re e Regina}

Su una scacchiera $8 \times 8$ sono posizionati due pezzi: il Re bianco e la Regina nera.  
Scrivere un programma che, acquisite le posizioni del Re e della Regina, determini se la Regina è in posizione tale da poter mangiare il Re.  

Le posizioni dei due pezzi sono identificate mediante la riga e la colonna su cui si trovano, espresse come numeri interi tra 1 e 8.

	
	
\subsection{Iterazioni}

\subsubsection{Primi $n$ numeri}
	Scrivere un programma che chiede all'utente di inserire un numero intero $n$ e stampa i primi $n$ numeri naturali a partire da 1.
	
	
\subsubsection{Somma di N numeri}
Scrivere un programma che chiede all'utente di inserire un numero intero $N$, che rappresenta la quantità di numeri da sommare. Successivamente il programma legge i $N$ numeri inseriti dall'utente e ne calcola la somma totale.

\subsubsection{Numero Positivo}
Scrivere un programma che chiede all'utente di inserire un numero intero positivo $n$. Se l'utente inserisce un numero negativo o nullo, il programma deve continuare a richiedere il valore finché non viene inserito un numero positivo.


\subsubsection{Calcolo della Media}
Scrivere un programma che calcoli la media aritmetica di una serie di numeri inseriti da tastiera.

\begin{itemize}
	\item Nella prima versione, l'inserimento termina quando l'utente digita il valore \texttt{0}. Il programma deve poi calcolare e stampare la media dei valori inseriti (escludendo lo zero).
	\item Nella seconda versione, il programma chiede prima quanti numeri l'utente intende inserire (\texttt{n}) e poi calcola la media dei numeri forniti.
\end{itemize}


\subsubsection{Fattoriale}
Scrivere un programma che acquisisca da tastiera un numero intero positivo \texttt{N} e calcoli il valore del suo fattoriale.

Il fattoriale di un numero è il prodotto di tutti i numeri compresi tra 1 ed \texttt{N}.  
Si ricorda che:
\[
N! = 1 \times 2 \times 3 \times \dots \times N
\]
Inoltre, per convenzione:
\[
0! = 1
\]

\subsubsection{Tabellina}
Scrivere un programma che chieda all'utente un numero intero positivo e stampi la sua tabellina fino a 10.


\subsubsection{Fibonacci}
Scrivere un programma che chieda all'utente un numero intero positivo N e stampi i primi N termini della serie di Fibonacci.
La serie di Fibonacci è una sequenza di numeri tali per cui ognuno di essi corrisponde alla somma dei due precedenti:

\[
1 \quad 2 \quad 3 \quad 5 \quad 8 \quad 13 \quad 21 \quad 34 \quad \dots
\]










\newpage

\subsection{Strutture Miste}
\subsubsection{Divisori di un numero}
Scrivere un programma che, letto un numero intero positivo, visualizzi tutti i suoi divisori.

Si ricorda che il resto della divisione si calcola con l’operatore \texttt{\%}.  
Ad esempio:  
\[
5 \% 2 = 1
\]
	
	
	\subsubsection{Numeri Primi}
	Scrivere un programma che legge un numero intero positivo e verifica se è un numero primo.
	
	Si ricorda che un numero primo è un numero maggiore di 1 che non ha altri divisori oltre a 1 e se stesso.
	
	
	
	\subsubsection{Gioco "Indovina un numero"}
	
	Si scriva un programma che permetta di giocare al gioco "Indovina un numero".  
	
	Un primo utente inserisce da tastiera un numero segreto compreso tra 1 e 100.  
	Il secondo utente deve indovinare il numero entro un massimo di 10 tentativi.  
	
	Ad ogni tentativo il programma stampa:
	- "Esatto!" se il numero è corretto,  
	- "Troppo alto" se il numero ipotizzato è maggiore di quello segreto,  
	- "Troppo basso" se il numero ipotizzato è minore di quello segreto.  
	
	Se il numero non viene indovinato entro 10 tentativi, il programma stampa "Hai perso".
	
	
	
	
	
	
	
	
	
	
	

	
\end{document}
